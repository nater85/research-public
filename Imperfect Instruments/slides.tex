%%%%%%%%%%%%%%%%%%%%%%%%%%%%%%%%%%%%%%%%%%%%%%%%%%%%%%%%%%%%%%%%%%%%%%%%%%%%%%%%%%%%%%%%%%%%%%%%%%%%%%
%%%%%%%%%%%%%%%%%%%%%%%%%%%%%%%%%%%%%%%%%%%%%%%%%%%%%%%%%%%%%%%%%%%%%%%%%%%%%%%%%%%%%%%%%%%%%%%%%%%%%%
\documentclass{beamer}
\usepackage{amsmath}               
\newtheorem{theo}{Theorem}
\newtheorem{coro}{Corollary}
\usetheme{Warsaw}
%\usecolortheme{crane}
\begin{document}
%%%%%%%%%%%%%%%%%%%%%%%%%%%%%%%%%%%%%%%%%%%%%%%%%%%%%%%%%%%%%%%%%%%%%%%%%%%%%%%%%%%%%%%%%%%%%%%%%%%%%%
%%%%%%%%%%%%%%%%%%%%%%%%%%%%%%%%%%%%%%%%%%%%%%%%%%%%%%%%%%%%%%%%%%%%%%%%%%%%%%%%%%%%%%%%%%%%%%%%%%%%%%

%%%%%%%%%%%%%%%%%%%%%%%%%%%%%%%%%%%%%%%%%%%%%%%%%%%%%%%%%%%%%%%%%%%%%%%%%%%%%%%%%%%%%%%%%%%%%%%%%%%%%%
%%%%%%%%%%%%%%%%%%%%%%%%%%%%%%%%%%%%%%%%%%%%%%%%%%%%%%%%%%%%%%%%%%%%%%%%%%%%%%%%%%%%%%%%%%%%%%%%%%%%%%

\title[Intransitivity in Correlations]{Tightening the Two-Sided Bounds of Nevo and Rosen when Instruments are Weak:}
\subtitle{An Application of the Property of Transitivity in Correlations}
	\author[S\o rensen Wiseman]
		{Todd S\o rensen \\ University of Nevada and IZA \\
		\medskip
		Nathan Wiseman \\ University of Nevada}
	\institute{Konstanz}
	\maketitle
%%%%%%%%%%%%%%%%%%%%%%%%%%%%%%%%%%%%%%%%%%%%%%%%%%%%%%%%%%%%%%%%%%%%%%%%%%%%%%%%%%%%%%%%%%%%%%%%%%%%%%
%%%%%%%%%%%%%%%%%%%%%%%%%%%%%%%%%%%%%%%%%%%%%%%%%%%%%%%%%%%%%%%%%%%%%%%%%%%%%%%%%%%%%%%%%%%%%%%%%%%%%%

%%%%%%%%%%%%%%%%%%%%%%%%%%%%%%%%%%%%%%%%%%%%%%%%%%%%%%%%%%%%%%%%%%%%%%%%%%%%%%%%%%%%%%%%%%%%%%%%%%%%%%
%%%%%%%%%%%%%%%%%%%%%%%%%%%%%%%%%%%%%%%%%%%%%%%%%%%%%%%%%%%%%%%%%%%%%%%%%%%%%%%%%%%%%%%%%%%%%%%%%%%%%%
\section{Background}
%%%%%%%%%%%%%%%%%%%%%%%%%%%%%%%%%%%%%%%%%%%%%%%%%%%%%%%%%%%%%%%%%%%%%%%%%%%%%%%%%%%%%%%%%%%%%%%%%%%%%%
%%%%%%%%%%%%%%%%%%%%%%%%%%%%%%%%%%%%%%%%%%%%%%%%%%%%%%%%%%%%%%%%%%%%%%%%%%%%%%%%%%%%%%%%%%%%%%%%%%%%%%

%%%%%%%%%%%%%%%%%%%%%%%%%%%%%%%%%%%%%%%%%%%%%%%%%%
\subsection{DGP}
%%%%%%%%%%%%%%%%%%%%%%%%%%%%%%%%%%%%%%%%%%%%%%%%%%

%%%%%%%%%%%%%%%%%%%%%%%%%%%%%%%%%%%%%%%%%%%%%%%%%%
\begin{frame}
\frametitle{DGP}
\textbf{DGP} \\ \medskip \pause

Consider the following data generating process: \\

\begin{equation*}
y=x\beta+u
\end{equation*}

\end{frame}
%%%%%%%%%%%%%%%%%%%%%%%%%%%%%%%%%%%%%%%%%%%%%%%%%%

%%%%%%%%%%%%%%%%%%%%%%%%%%%%%%%%%%%%%%%%%%%%%%%%%%
\begin{frame}
\frametitle{OLS}
\textbf{OLS} \\ \medskip  \pause

If $x$ is correlated with $u$, of course OLS will be biased: \pause

\begin{eqnarray*}
plim[\beta_{OLS}]&=&plim[(x'x)^{-1}(x'y)] \\ \pause
&=&plim[(x'x)^{-1}(x'(x\beta+u))] \\ \pause
&=&(x'x)^{-1}(x'x)\beta+plim[(x'x)^{-1}(x'u)] \\ \pause
&=&\beta+\frac{\sigma_{xu}}{\sigma_x^2} \pause
\end{eqnarray*}

And the asymptotic bias will be \pause

\begin{equation*}
\beta_{OLS}-\beta=\frac{\sigma_{xu}}{\sigma_x^2} 
\end{equation*}

\end{frame}
%%%%%%%%%%%%%%%%%%%%%%%%%%%%%%%%%%%%%%%%%%%%%%%%%%

%%%%%%%%%%%%%%%%%%%%%%%%%%%%%%%%%%%%%%%%%%%%%%%%%%
\begin{frame}
\frametitle{IV}
\textbf{IV} \\ \medskip  \pause

One solution is to find some $z$ s.t. $\rho_{xz}\neq0$ and use IV \pause

\begin{equation*}
\beta_{IV}=(z'x)^{-1}(z'y) \pause
\end{equation*}

We typically would assume that $\rho_{zu}=0$ and move on, \\ \medskip \pause but here we will think of IV causing problems as well when this fails:

\end{frame}
%%%%%%%%%%%%%%%%%%%%%%%%%%%%%%%%%%%%%%%%%%%%%%%%%%

%%%%%%%%%%%%%%%%%%%%%%%%%%%%%%%%%%%%%%%%%%%%%%%%%%
\begin{frame}
\frametitle{IV}
\textbf{IV} \\ \medskip  \pause

\begin{eqnarray*}
plim[\beta_{IV}]&=&plim[(z'x)^{-1}(z'y)] \\ \pause
&=&plim[(z'x)^{-1}(z'(x\beta+u))] \\  \pause
&=&(z'x)^{-1}(z'x)\beta+plim(z'x)^{-1}(z'u) \\  \pause
&=&\beta+\frac{\sigma_{zu}}{\sigma_{xz}}  \pause
\end{eqnarray*}

And the asymptotic bias will be \pause

\begin{equation*}
\beta_{IV}-\beta=\frac{\sigma_{zu}}{\sigma_{xz}}
\end{equation*}

\end{frame}
%%%%%%%%%%%%%%%%%%%%%%%%%%%%%%%%%%%%%%%%%%%%%%%%%%

%%%%%%%%%%%%%%%%%%%%%%%%%%%%%%%%%%%%%%%%%%%%%%%%%%
\begin{frame}
\frametitle{Bias}

\begin{eqnarray*}
\beta_{OLS}-\beta&=&\frac{\sigma_{xu}}{\sigma_x^2} \\
\beta_{IV}-\beta&=&\frac{\sigma_{zu}}{\sigma_{xz}} \pause
\end{eqnarray*}

\textbf{4 Cases}: \pause Assume $\rho_{xu}\rho_{zu}>0$ \\ \pause (will always hold as we can redefine $z$ as $z=-w$). \pause


\begin{enumerate}
  \item $\rho_{xu}>0$, $\rho_{zu}>0$, $\rho_{xz}>0$: \pause OLS is biased upward, IV is biased upward.  \pause
  \item $\rho_{xu}>0$, $\rho_{zu}>0$, $\rho_{xz}<0$: \pause OLS is biased upward, IV is biased downwards.  \pause
  \item $\rho_{xu}<0$, $\rho_{zu}<0$, $\rho_{xz}>0:$ \pause OLS is biased downward, IV is biased downwards.  \pause
  \item $\rho_{xu}<0$, $\rho_{zu}<0$, $\rho_{xz}<0:$ \pause OLS is biased downward, IV is biased upward.
\end{enumerate}

\end{frame}
%%%%%%%%%%%%%%%%%%%%%%%%%%%%%%%%%%%%%%%%%%%%%%%%%%

%%%%%%%%%%%%%%%%%%%%%%%%%%%%%%%%%%%%%%%%%%%%%%%%%%
\begin{frame}
\frametitle{Bounding}
\textbf{Bounding} \\ \medskip \pause

Above biases informs us as to how OLS and IV provide bounds: \\ \pause

\begin{eqnarray}
  &\beta&<min\{\beta_{OLS},\beta_{IV}\} \\ \pause 
  \beta_{IV}<&\beta&<\beta_{OLS}  \\ \pause 
  max\{\beta_{OLS},\beta_{IV}\}<&\beta& \\ \pause
  \beta_{OLS}<&\beta&<\beta_{IV}
\end{eqnarray}

\end{frame}
%%%%%%%%%%%%%%%%%%%%%%%%%%%%%%%%%%%%%%%%%%%%%%%%%%

%%%%%%%%%%%%%%%%%%%%%%%%%%%%%%%%%%%%%%%%%%%%%%%%%%
\subsection{Nevo and Rosen}
%%%%%%%%%%%%%%%%%%%%%%%%%%%%%%%%%%%%%%%%%%%%%%%%%%

%%%%%%%%%%%%%%%%%%%%%%%%%%%%%%%%%%%%%%%%%%%%%%%%%%
\begin{frame}
\frametitle{New IV} 
\textbf{New IV} \\ \medskip \pause

Nevo and Rosen expand on this with a new instrument: \pause

\begin{equation*}
V(\lambda)=\sigma_{x}z-\lambda\sigma_{z}x \pause
\end{equation*}

\begin{itemize}
\item Bias disappears completely when $\lambda=\lambda^*=\frac{\rho_{zu}}{\rho_{xu}}$ \pause
\item This term is unknown \pause
\item NR take limiting case $\lambda=1$: your IV is no worse than X itself \pause
\item Even in this case, the IV ``V(1)'' improves upon OLS
\end{itemize}

\end{frame}
%%%%%%%%%%%%%%%%%%%%%%%%%%%%%%%%%%%%%%%%%%%%%%%%%%

%%%%%%%%%%%%%%%%%%%%%%%%%%%%%%%%%%%%%%%%%%%%%%%%%%
\begin{frame}
\frametitle{New Bounds}
\textbf{New Bounds: One of the Two Bounds is Improved Upon} \\ \medskip \pause

\setcounter{equation}{0}
\begin{eqnarray}
     &\beta&<min\{\beta_{IV}^V,\beta_{IV}^Z\} \\  \pause
     \beta_{IV}^Z<&\beta&<\beta_{IV}^V \\ \pause
     max\{\beta_{IV}^V,\beta_{IV}^Z\}<&\beta& \\ \pause
     \beta_{IV}^V<&\beta&<\beta_{IV}^Z
\end{eqnarray}

\end{frame}
%%%%%%%%%%%%%%%%%%%%%%%%%%%%%%%%%%%%%%%%%%%%%%%%%%

%%%%%%%%%%%%%%%%%%%%%%%%%%%%%%%%%%%%%%%%%%%%%%%%%%%%%%%%%%%%%%%%%%%%%%%%%%%%%%%%%%%%%%%%%%%%%%%%%%%%%%
%%%%%%%%%%%%%%%%%%%%%%%%%%%%%%%%%%%%%%%%%%%%%%%%%%%%%%%%%%%%%%%%%%%%%%%%%%%%%%%%%%%%%%%%%%%%%%%%%%%%%%
\section{New Bounds}
%%%%%%%%%%%%%%%%%%%%%%%%%%%%%%%%%%%%%%%%%%%%%%%%%%%%%%%%%%%%%%%%%%%%%%%%%%%%%%%%%%%%%%%%%%%%%%%%%%%%%%
%%%%%%%%%%%%%%%%%%%%%%%%%%%%%%%%%%%%%%%%%%%%%%%%%%%%%%%%%%%%%%%%%%%%%%%%%%%%%%%%%%%%%%%%%%%%%%%%%%%%%%

%%%%%%%%%%%%%%%%%%%%%%%%%%%%%%%%%%%%%%%%%%%%%%%%%%
\subsection{Contribution}
%%%%%%%%%%%%%%%%%%%%%%%%%%%%%%%%%%%%%%%%%%%%%%%%%%

%%%%%%%%%%%%%%%%%%%%%%%%%%%%%%%%%%%%%%%%%%%%%%%%%%
\begin{frame}
\frametitle{Our Contribution}
\textbf{We Improve Upon these Bounds Further} \\ \medskip \pause

\begin{itemize}
  \item We make a complementary contribution in the cases of two sided bounds (2 and 4) \pause
  \item We provided a new bound that will in some cases improve upon IV Z \pause
  \item We do this by leveraging the assumptions in this case to introduce a new concept to the econometrics literature: \pause \pause
  \item \textbf{Transitivity in Correlations} \pause
  \item Our simulations show that this leads to improved bound and as the IV becomes weaker and more correlated with the unobservables
\end{itemize}

\end{frame}
%%%%%%%%%%%%%%%%%%%%%%%%%%%%%%%%%%%%%%%%%%%%%%%%%%

%%%%%%%%%%%%%%%%%%%%%%%%%%%%%%%%%%%%%%%%%%%%%%%%%%
\subsection{Transitivity in Correlations}
%%%%%%%%%%%%%%%%%%%%%%%%%%%%%%%%%%%%%%%%%%%%%%%%%%

%%%%%%%%%%%%%%%%%%%%%%%%%%%%%%%%%%%%%%%%%%%%%%%%%%
\begin{frame}
\frametitle{Transitivity in Correlations}
\textbf{Transitivity in Correlations} \\ \bigskip \pause 

Two publications in the Statistics Literature establish this property. \\ \medskip \pause

We use these properties, which require no additional assumptions, to derive a new set of bounds. \\ \medskip 

\end{frame}
%%%%%%%%%%%%%%%%%%%%%%%%%%%%%%%%%%%%%%%%%%%%%%%%%%

%%%%%%%%%%%%%%%%%%%%%%%%%%%%%%%%%%%%%%%%%%%%%%%%%%
\begin{frame}
\frametitle{Transitivity in Correlations}
\frametitle{Intuition}
\textbf{Intuition} \\ \bigskip \pause 

\textbf{Consider Case 2} \pause
\begin{itemize}
  \item $\rho_{xu}>0$, $\rho_{zu}>0$, $\rho_{xz}<0$ \pause
  \item $u$ is positively correlated with $x$ \pause 
  \item $x$ is negatively correlated with $z$ \pause 
  \item \emph{transitivity} leads us to believe that $u$ will be negatively correlated with $z$ \pause
  \item But $\rho_{zu}>0$ in this case \pause
  \item So this is a case of \emph{intransitive correlations}
\end{itemize}

\end{frame}
%%%%%%%%%%%%%%%%%%%%%%%%%%%%%%%%%%%%%%%%%%%%%%%%%%

%%%%%%%%%%%%%%%%%%%%%%%%%%%%%%%%%%%%%%%%%%%%%%%%%%
\begin{frame}
\frametitle{Transitivity in Correlations}
\frametitle{Intuition}
\textbf{Intuition} \\ \bigskip \pause 

\textbf{Consider Case 4}
\begin{itemize}
  \item $\rho_{xu}<0$, $\rho_{zu}<0$, $\rho_{xz}<0:$ \pause 
  \item $u$ is negatively correlated with $x$ \pause 
  \item $x$ is negatively correlated with $z$ \pause 
  \item \emph{transitivity} leads us to believe that $u$ will be positively correlated with $z$ \pause 
  \item But $\rho_{zu}<0$ in this case \pause 
  \item So this is also a case of \emph{intransitive correlations} 
\end{itemize}

\end{frame}
%%%%%%%%%%%%%%%%%%%%%%%%%%%%%%%%%%%%%%%%%%%%%%%%%%

%%%%%%%%%%%%%%%%%%%%%%%%%%%%%%%%%%%%%%%%%%%%%%%%%%
\begin{frame}
\frametitle{Transitivity in Correlations}
\framesubtitle{Theoretical Background}
\textbf{Theorem 1} \\ \bigskip \pause

Langford et al (2001, \emph{American Statistician}) introduces the concept of transitivity in correlations: \bigskip \pause

\begin{theo} \label{theo:trans_1}
A sufficient condition for positive correlation between $A$ and $C$ ($\rho_{AC}>0$), when $\rho_{AB}\rho_{BC}>0$ can be stated as follows: $\rho_{AB}^2+\rho_{BC}^2>1 \implies \rho_{AC}>0$.  \pause
\end{theo}

Or: if two $\rho$s are big enough, the 3rd one must go the way we expected!

\end{frame}
%%%%%%%%%%%%%%%%%%%%%%%%%%%%%%%%%%%%%%%%%%%%%%%%%%

%%%%%%%%%%%%%%%%%%%%%%%%%%%%%%%%%%%%%%%%%%%%%%%%%%
\begin{frame}
\frametitle{Transitivity in Correlations}
\framesubtitle{Theoretical Background}
\textbf{Corollary 1} \\ \bigskip \pause

We provide the following corollary of \emph{intransitivity in correlations}: \bigskip \pause

\begin{coro} \label{coro:trans_1}
A necessary condition for negative correlation between $A$ and $C$ ($\rho_{AC}<0$), when $\rho_{AB}\rho_{BC}>0$, is as follows: $\rho_{AC} <0 \implies \rho_{AB}^2+\rho_{BC}^2 < 1$. \pause
\end{coro}

Or: if the 3rd one didn't go the way we expected, the first two $\rho$s must not have been big enough!
\end{frame}
%%%%%%%%%%%%%%%%%%%%%%%%%%%%%%%%%%%%%%%%%%%%%%%%%%

%%%%%%%%%%%%%%%%%%%%%%%%%%%%%%%%%%%%%%%%%%%%%%%%%%
\begin{frame}
\frametitle{Transitivity in Correlations}
\framesubtitle{Theoretical Background}
\textbf{Theorem 2} \\ \bigskip \pause

Lepovotsky and Conklin (2004) extend this research with a second theorem \pause

\begin{theo} \label{theo:trans_2}
A sufficient condition for negative correlation between $A$ and $C$ ($\rho_{AC}<0$), when $\rho_{AB}\rho_{BC}<0$, can be stated as follows: $\rho_{AB}^2+\rho_{BC}^2>1 \implies \rho_{AC}<0 $ \pause
\end{theo}

Same intuition, just in the ``negative'' case.

\end{frame}
%%%%%%%%%%%%%%%%%%%%%%%%%%%%%%%%%%%%%%%%%%%%%%%%%%

%%%%%%%%%%%%%%%%%%%%%%%%%%%%%%%%%%%%%%%%%%%%%%%%%%
\begin{frame}
\frametitle{Transitivity in Correlations}
\framesubtitle{Theoretical Background}
\textbf{Corollary 2} \\ \bigskip \pause

We again provide a corollary: \bigskip \pause 

\begin{coro} \label{coro:trans_2}
A necessary condition for positive correlation between $A$ and $C$ ($\rho_{AC}>0$), when $\rho_{AB}\rho_{BC}<0$, is as follows: $\rho_{AC} >0 \implies \rho_{AB}^2+\rho_{BC}^2 < 1$. 
\end{coro}

\end{frame}
%%%%%%%%%%%%%%%%%%%%%%%%%%%%%%%%%%%%%%%%%%%%%%%%%%

%%%%%%%%%%%%%%%%%%%%%%%%%%%%%%%%%%%%%%%%%%%%%%%%%%
\begin{frame}
\frametitle{Transitivity in Correlations}
\framesubtitle{Application}
\textbf{Application} \\ \bigskip \pause

To Case 4: \pause

\begin{itemize}
  \item $\rho_{xu}<0$, $\rho_{zu}<0$, $\rho_{xz}<0$ \pause
  \item $\rho_{xz}\rho_{xu}>0$ \pause
  \item So we use Corrolary 1  \pause
  \item Correspondence A=Z, B=X, C=U tells us that  \pause
  \item $\rho_{zu}<0$ \pause implies $\rho_{xz}^2+\rho_{xu}^2<1$ \pause
  \item Because we would expect $\rho_{zu}>0$, so there must not be ``\emph{too much}'' information in the other two $\rho$ terms.
\end{itemize}

\end{frame}
%%%%%%%%%%%%%%%%%%%%%%%%%%%%%%%%%%%%%%%%%%%%%%%%%%

%%%%%%%%%%%%%%%%%%%%%%%%%%%%%%%%%%%%%%%%%%%%%%%%%%
\begin{frame}
\frametitle{Transitivity in Correlations}
\framesubtitle{Application}
\textbf{Application} \\ \bigskip \pause

To Case 2: \pause

\begin{itemize}
  \item $\rho_{xu}>0$, $\rho_{zu}>0$, $\rho_{xz}<0$ \pause
  \item $\rho_{xz}\rho_{xu}<0$ \pause
  \item So we use Corollary 2 \pause
  \item Correspondence  A=Z, B=X, C=U tells us that  \pause
  \item $\rho_{zu}<0$ \pause implies $\rho_{xz}^2+\rho_{xu}^2<1$  \pause
  \item Because we would expect $\rho_{zu}<0$, so there must not be ``\emph{too much}'' information in the other two $\rho$ terms.
\end{itemize}

\end{frame}
%%%%%%%%%%%%%%%%%%%%%%%%%%%%%%%%%%%%%%%%%%%%%%%%%%

%%%%%%%%%%%%%%%%%%%%%%%%%%%%%%%%%%%%%%%%%%%%%%%%%%
\subsection{Derivation}
%%%%%%%%%%%%%%%%%%%%%%%%%%%%%%%%%%%%%%%%%%%%%%%%%%

%%%%%%%%%%%%%%%%%%%%%%%%%%%%%%%%%%%%%%%%%%%%%%%%%%
\begin{frame}
\frametitle{Derivation of New Bounds}
\textbf{Derivation of New Bounds } \\ \bigskip \pause

In both cases, we work from \pause

\begin{equation*}
\rho_{xu}^2+\rho_{xz}^2<1 \label{eqn:cond_3} \pause
\end{equation*}

Solving a quadratic inequality for $\rho_{xu}$ \pause

\begin{eqnarray*}
\rho_{xu}^2 &<&{1-\rho_{xz}^2} \nonumber \\ \pause
\frac{\sigma_{xu}}{\sigma_x \sigma_u}=\rho_{xu}&\in& \pm \sqrt{1-\rho_{xz}^2} \nonumber \\
\end{eqnarray*}

\end{frame}
%%%%%%%%%%%%%%%%%%%%%%%%%%%%%%%%%%%%%%%%%%%%%%%%%%

%%%%%%%%%%%%%%%%%%%%%%%%%%%%%%%%%%%%%%%%%%%%%%%%%%
\begin{frame}
\frametitle{Derivation of New Bounds}
\textbf{Derivation of New Bounds } \\ \bigskip \pause

Multiplying by $\frac{\sigma_u}{\sigma_x}$ \pause

\begin{eqnarray}
\frac{\sigma_{xu}}{\sigma_x^2}=(\beta_{OLS}-\beta)&\in& \pm \frac{\sigma_u}{\sigma_x} \sqrt{1-\rho_{xz}^2} \nonumber 
\end{eqnarray}

\end{frame}
%%%%%%%%%%%%%%%%%%%%%%%%%%%%%%%%%%%%%%%%%%%%%%%%%%

%%%%%%%%%%%%%%%%%%%%%%%%%%%%%%%%%%%%%%%%%%%%%%%%%%
\begin{frame}
\frametitle{Derivation of New Bounds}
\textbf{Derivation of New Bounds } \\ \bigskip \pause

Shifting the set by $-\beta_{OLS}$ and then transforming it by a negative we have two new bounds:  \pause
\begin{eqnarray*}
-\beta&\in& \pm \frac{\sigma_u}{\sigma_x} \sqrt{1-\rho_{xz}^2}-\beta_{OLS} \nonumber \\ \pause
\beta&\in&\beta_{OLS}\pm \frac{\sigma_u}{\sigma_x}\sqrt{1-\rho_{xz}^2} \\ \pause
\beta&\in&(\beta_{L},\beta_{U}) \label{eqn:bounds}
\end{eqnarray*}

\end{frame}
%%%%%%%%%%%%%%%%%%%%%%%%%%%%%%%%%%%%%%%%%%%%%%%%%%

%%%%%%%%%%%%%%%%%%%%%%%%%%%%%%%%%%%%%%%%%%%%%%%%%%
\subsection{Evaluating New Bounds}
%%%%%%%%%%%%%%%%%%%%%%%%%%%%%%%%%%%%%%%%%%%%%%%%%%

%%%%%%%%%%%%%%%%%%%%%%%%%%%%%%%%%%%%%%%%%%%%%%%%%%
\begin{frame}
\frametitle{Evaluating New Bounds}
\textbf{Evaluating New Bounds} \\ \bigskip \pause

In Case 2 (4), OLS was an upper (lower) bound dominated by NR's new bound. \\ \pause
So in Case 2, we are interested in $\beta_{L}$, and in Case 4 in $\beta_{H}$. \\ \pause
We can then express the bounds as  \\ \pause

\begin{eqnarray*}
     &\beta&<min\{\beta_{IV}^V,\beta_{IV}^Z\} \\  \pause
     max(\beta_{IV}^Z,\beta_L)<&\beta&<\beta_{IV}^V \\ \pause
     max\{\beta_{IV}^V,\beta_{IV}^Z\}<&\beta& \\ \pause
     \beta_{IV}^V<&\beta&<min(\beta_{IV}^Z,\beta_H)
\end{eqnarray*}

\end{frame}
%%%%%%%%%%%%%%%%%%%%%%%%%%%%%%%%%%%%%%%%%%%%%%%%%%

%%%%%%%%%%%%%%%%%%%%%%%%%%%%%%%%%%%%%%%%%%%%%%%%%%
\begin{frame}
\frametitle{Evaluating New Bounds}
\textbf{Evaluating New Bounds} \\ \bigskip \pause

\begin{itemize}
  \item Let's focus on Case 2. \\ \pause
  \item We might do better then IV: \\ \pause
  \item This bound will be tighter as $\rho_{xz}$ become smaller \pause
  \item And also as everything goes to hell with regular IV ($\rho_{xz}$ smaller, $\rho_{zu}$ bigger) \pause
  \item We now turn to simulations
\end{itemize}

\end{frame}
%%%%%%%%%%%%%%%%%%%%%%%%%%%%%%%%%%%%%%%%%%%%%%%%%%

%%%%%%%%%%%%%%%%%%%%%%%%%%%%%%%%%%%%%%%%%%%%%%%%%%
\section{Simulations}
%%%%%%%%%%%%%%%%%%%%%%%%%%%%%%%%%%%%%%%%%%%%%%%%%%

%%%%%%%%%%%%%%%%%%%%%%%%%%%%%%%%%%%%%%%%%%%%%%%%%%
\begin{frame}
\frametitle{Simulation}
\framesubtitle{DGP}
\textbf{DGP} \\ \bigskip \pause

\begin{itemize}
     \item $y=.5x+u$ \pause
     \item $x=e_1-\alpha_1\times c+\alpha_2 \times u$ \pause
     \item $z=e_2-\alpha_3\times c+\alpha_4 \times u$ \pause
     \item $\alpha_1$, $\alpha_3$ are uniform(0,2) \pause
     \item $\alpha_2$, $\alpha_4$ are uniform(0,1) \pause
     \item Other terms are N(0,1)
\end{itemize}

\end{frame}
%%%%%%%%%%%%%%%%%%%%%%%%%%%%%%%%%%%%%%%%%%%%%%%%%%

%%%%%%%%%%%%%%%%%%%%%%%%%%%%%%%%%%%%%%%%%%%%%%%%%%
\begin{frame}
\frametitle{Simulation}
\framesubtitle{Simulations}
\textbf{Simulations} \\ \bigskip \pause

\begin{itemize}
     \item Dataset: N=1,000,000 \pause
     \item 100,000 Simulations \pause
     \item Tables Below \pause
     \item Columns: midpoint of $\lambda=\frac{\rho_{zu}}{\rho_{xu}}$ bin \pause
     \item Rows: midpoint of $\rho_{xz}$ bin
\end{itemize}

\end{frame}
%%%%%%%%%%%%%%%%%%%%%%%%%%%%%%%%%%%%%%%%%%%%%%%%%%

%%%%%%%%%%%%%%%%%%%%%%%%%%%%%%%%%%%%%%%%%%%%%%%%%%
\begin{frame}
\frametitle{Simulation}
\framesubtitle{Counts}
\textbf{Number of Iterations in Each Bin} \\ \pause
\footnotesize{
\input{count}
}
\end{frame}
%%%%%%%%%%%%%%%%%%%%%%%%%%%%%%%%%%%%%%%%%%%%%%%%%%

%%%%%%%%%%%%%%%%%%%%%%%%%%%%%%%%%%%%%%%%%%%%%%%%%%
\begin{frame}
\frametitle{Simulation}
\framesubtitle{IV}
\textbf{IV} \\ \pause
\footnotesize{
\input{iv}
}
\end{frame}
%%%%%%%%%%%%%%%%%%%%%%%%%%%%%%%%%%%%%%%%%%%%%%%%%%

%%%%%%%%%%%%%%%%%%%%%%%%%%%%%%%%%%%%%%%%%%%%%%%%%%
\begin{frame}
\frametitle{Simulation}
\framesubtitle{Our Lower Bound}
\textbf{Our Lower Bound} \\ \pause
\footnotesize{
\input{lower}
}
\end{frame}
%%%%%%%%%%%%%%%%%%%%%%%%%%%%%%%%%%%%%%%%%%%%%%%%%%

%%%%%%%%%%%%%%%%%%%%%%%%%%%%%%%%%%%%%%%%%%%%%%%%%%
\begin{frame}
\frametitle{Simulation}
\framesubtitle{Share with Improvement}
\textbf{Share with Improvement} \\ \pause
\footnotesize{
\input{better}
}
\end{frame}
%%%%%%%%%%%%%%%%%%%%%%%%%%%%%%%%%%%%%%%%%%%%%%%%%%

%%%%%%%%%%%%%%%%%%%%%%%%%%%%%%%%%%%%%%%%%%%%%%%%%%
\section{Conclusions}
%%%%%%%%%%%%%%%%%%%%%%%%%%%%%%%%%%%%%%%%%%%%%%%%%%

%%%%%%%%%%%%%%%%%%%%%%%%%%%%%%%%%%%%%%%%%%%%%%%%%%
\begin{frame}
\frametitle{Conclusions}
\framesubtitle{Summary of Earlier Work}
\textbf{Earlier Work} \\ \bigskip \pause

\begin{itemize}
  \item When we assume direction of bias, IV and OLS provide us with bounds \pause
  \item These are two sided in two of four cases \pause
  \item Nevo and Rosen improve either lower or upper bound in this case
\end{itemize}

\end{frame}
%%%%%%%%%%%%%%%%%%%%%%%%%%%%%%%%%%%%%%%%%%%%%%%%%%

%%%%%%%%%%%%%%%%%%%%%%%%%%%%%%%%%%%%%%%%%%%%%%%%%%
\begin{frame}
\frametitle{Conclusions}
\framesubtitle{Our Contribution}
\textbf{Our Contribution} \\ \bigskip \pause

\begin{itemize}
  \item We introduce the concept of \emph{transitivity in correlations} to the econometrics literature for the first time \pause
  \item We show that the two sided bounds represent \emph{intransitive correlations} \pause
  \item This provides us with new information \pause
  \item The new information provides us with new bounds, which may improve upon the bound NR did not improve upon \pause
  \item Improvement most likely when IV is more invalid or weaker \pause
\end{itemize}

\end{frame}
%%%%%%%%%%%%%%%%%%%%%%%%%%%%%%%%%%%%%%%%%%%%%%%%%%

\end{document}


